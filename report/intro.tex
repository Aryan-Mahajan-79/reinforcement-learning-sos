\section{Introduction}
\begin{center}
    \textbf{\Huge{Edit it Later!!!}}
\end{center}

As I embarked on my journey into the realm of machine learning, one area that particularly captured my interest was reinforcement learning (RL). This subset of machine learning stands out because of its unique approach to training algorithms through interaction with an environment. Unlike supervised learning, which relies on a vast dataset of labeled examples, reinforcement learning focuses on learning optimal behaviors through trial and error.

In reinforcement learning, an agent learns to make decisions by performing actions in a given environment to achieve a specific goal. To draw a simple analogy, consider playing a video game. Here, the agent represents the player, and the environment corresponds to the game world. The player's objective is to make moves that score points or advance levels. Through continuous interaction with the game, the player gradually learns which actions yield the highest scores or successfully complete the levels.

This learning process is driven by the concept of rewards and punishments. The agent receives feedback from the environment in the form of rewards (positive feedback) or penalties (negative feedback) based on its actions. Over time, the agent aims to maximize its cumulative reward by identifying and executing actions that yield the most favorable outcomes. This method of learning closely mimics the way humans and animals learn from their surroundings, making it a compelling area of study.

Reinforcement learning has already demonstrated its potential in various domains, from game playing (such as AlphaGo and OpenAI's Dota 2 bot) to robotics, autonomous driving, and even financial trading. Its ability to adapt and improve through interaction makes it a powerful tool for tackling complex decision-making problems.

As I continue to explore this fascinating field, I am eager to delve deeper into the algorithms and techniques that drive reinforcement learning. Understanding these foundational concepts is the first step towards leveraging RL to develop intelligent systems capable of making informed and autonomous decisions.